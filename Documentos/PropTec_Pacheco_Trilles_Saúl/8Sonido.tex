\chapter{Sonido}
Para crear el sonido de un videojuego hay que dotarlo de una identidad, nos tenemos que asegurar de que los jugadores reconozcan la melodía simplemente con unas cuantas notas, para esto necesitamos un estribillo pegadizo y que se repita un número de veces abundante, no excesivo para que no llegue a cansar pero para que la gente lo recuerde bien.


\subsection*{Diseño de Sonido Ambiente}
El sonido ambiente del juego está diseñado para crear una atmósfera envolvente que complementa la jugabilidad. Queremos que el jugador sienta que está en un mundo vivo lleno de fantasía. Cada área tendrá una intensidad sonora distinta, para conseguir esto utilizaremos efectos ambientales y música adaptativa para reflejar el entorno y las acciones del jugador.

La música ambiental y los sonidos de mecánicas van a sonar con menos
volumen que el resto del soundtrack.

En el apartado de extra de la siguiente lista se encuentran, los efectos y estructuras que se han de seguir para crear la música ambiental respecto a cada continente.

\paragraph{Continentes}
\begin{itemize}
    \item \textbf{Thalvaren} \newline
    \textbf{Cómo es}: Temas con un ritmo más marcado y crescendos constantes. Se utilizan metales (trompetas, trombones) y coros angelicales que acompañan la sensación de logro.\newline
    \textbf{Qué transmite}: Valentía, determinación y la sensación de estar en una gran misión.\newline
    \textbf{Ejemplo emocional}: Marchar hacia una batalla épica con un ejército detrás o llegar a una ciudad majestuosa por primera vez.\newline
    \textbf{Extra}: \href{https://www.youtube.com/watch?v=7LcTbUk0D5Q&ab_channel=Ludofon%C3%ADa}{\emph{Música fantasía modo Lidio}} y la música podría inspirarse en \href{https://www.youtube.com/watch?v=lkpG-_v0yPc&ab_channel=PeterCrowley%27sFantasyDream}{\emph{Instrumental de Peter Crowley}}
    
    \item \textbf{Durmon} \newline
    \textbf{Cómo es}: Ritmos intensos y repetitivos, con sonidos más oscuros, bajos profundos y percusiones contundentes. Se puede sentir discordancia en las notas o cambios de tono bruscos.\newline
    \textbf{Qué transmite}: Ansiedad, inseguridad, alerta o miedo.\newline
    \textbf{Ejemplo emocional}: Encontrarte cara a cara con una criatura poderosa o entrar en un castillo abandonado donde el peligro parece acechar en cada sombra.\newline
    \textbf{Extra}: \href{https://www.youtube.com/watch?v=CCfNsBHM3fg&ab_channel=Ludofon%C3%ADa}{\emph{Música en niveles selváticos}} y la música podría inspirarse en \href{https://www.youtube.com/watch?v=uzMlDOm6V3Y&ab_channel=PeterCrowley%27sFantasyDream}{\emph{Instrumental de Peter Crowley}}
    
    \item \textbf{Iryndor} \newline
    \textbf{Cómo es}: Ritmos alegres, ligeros y fugaces, con melodías efímeras. Usan flautas, tambores pequeños y cuerdas animadas (número reducido).\newline
    \textbf{Qué transmite}: Alegría, diversión o despreocupación.\newline
    \textbf{Ejemplo emocional}: Participar en un festival en una aldea mágica o ver a personajes celebrar una victoria.\newline
    \textbf{Extra}: \href{https://youtu.be/Havwy8-1HNc?si=w1-2gOwLiPOCpVeh}{\emph{Música en niveles desérticos}}
    
    
    \item \textbf{Nytheria} \newline
    \textbf{Cómo es}: Melodías etéreas y flotantes con ecos, sintetizadores suaves. Usualmente en tonalidades menores, con cambios sutiles.\newline
    \textbf{Qué transmite}: Intriga, misterio, asombro, lo desconocido o lo sobrenatural.\newline
    \textbf{Ejemplo emocional}: Explorar ruinas subterráneas antiguas con grabados en las paredes o perderte por unas grutas sobrenaturales.\newline
    \textbf{Extra}: \href{https://www.youtube.com/watch?v=01o0p9bv1hA&ab_channel=Ludofon%C3%ADa}{\emph{Música tétrica modo \textit{glissando}}}

    \item \textbf{Kaedhyr} \newline
    \textbf{Cómo es}: Ritmos rápidos y dinámicos, con percusiones vivas y melodías ascendentes. Instrumentos como violines rápidos, tambores taiko o guitarras.\newline
    \textbf{Qué transmite}: Adrenalina, emoción, urgencia o pasión.\newline
    \textbf{Ejemplo emocional}: Una persecución frenética, un combate ágil o unos engranajes girando al ritmo del tick, tack.\newline
    \textbf{Extra}: \href{https://www.youtube.com/watch?v=B1p4ySnKAIw&ab_channel=Ludofon%C3%ADa}{\emph{Música desafiante modo frigio}}
\end{itemize}


\subsection*{Diseño de Sonido Mecánicas}

\begin{itemize}
    \item \textbf{Ataque cuerpo a cuerpo}: \href{https://freesound.org/people/Merrick079/sounds/566434/}{Sonido de ataque}
    \item \textbf{Ataque cuerpo a cuerpo con arma}: \href{https://freesound.org/people/Merrick079/sounds/568170/}{Sonido de ataque}
    \item \textbf{Ataque a distancia}: \href{https://freesound.org/people/checholio/sounds/443832/}{Sonido de disparo}
    \item \textbf{Uso de habilidades especiales}: \href{https://freesound.org/people/Rob_Marion/sounds/542039/}{Sonido de habilidad mágica}
    \item \textbf{Recoger objetos}: \href{https://freesound.org/people/Rob_Marion/sounds/542033/}{Sonido de recoger}
    \item \textbf{Interacción con el entorno}: \href{https://freesound.org/people/RanneM/sounds/475557/}{Sonido de interacción}
    \item \textbf{Saltar}: \href{https://freesound.org/people/Ziggler_Games/sounds/464336/}{Sonido de salto}
    \item \textbf{Caer}: \href{https://freesound.org/people/benjaminharve/sounds/420049/}{Sonido de caída}
    \item \textbf{Caminar}: \href{https://freesound.org/people/straget/sounds/411206/}{Sonido de caminar}
    \item \textbf{Correr}: \href{https://freesound.org/people/Slave2theLight/sounds/157037/}{Sonido de correr}
    \item \textbf{Muerte o derrota}: \href{https://freesound.org/people/benjaminharve/sounds/420048/}{Sonido de derrota}
    \item \textbf{Celebración o victoria}: \href{https://freesound.org/people/benjaminharve/sounds/420047/}{Sonido de victoria}
    \item \textbf{Abrir un cofre}: \href{https://freesound.org/people/Rob_Marion/sounds/542033/}{Sonido de abrir cofre}
    \item \textbf{Usar un objeto}: \href{https://freesound.org/people/benjaminharve/sounds/420045/}{Sonido de uso de objeto}
    \item \textbf{Esquivar}: \href{https://freesound.org/people/ArTiX.0/sounds/742717/}{Sonido de esquivar}
    \item \textbf{Rugido de enemigo}: \href{https://freesound.org/people/Merrick079/sounds/567892/}{Sonido de rugido}
    \item \textbf{Sonido de hechizo}: \href{https://freesound.org/people/Iridiuss/sounds/519414/}{Sonido de hechizo}
    \item \textbf{Sonido de puerta abriéndose}: \href{https://freesound.org/people/Rob_Marion/sounds/542033/}{Sonido de puerta}
\end{itemize}

