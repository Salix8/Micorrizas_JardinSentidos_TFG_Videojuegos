\chapter{Gamificación}

\subsection*{Estrategias}
Las estrategias de gamificación consisten en, métodos que incentivan la interacción del jugador con el entorno y fomentan un compromiso continuo. Por lo en esta lista vamos a recopilar dichos métodos en los que intentamos que el jugador se quede en nuestro juego.
\begin{itemize}
    \item \textbf{Sistema de recompensas}: 
    \begin{itemize}
        \item \textit{Loot boxes}: unas cajas que compras por gemas, y te puede tocar desde un 25\% de las gemas que has invertido por ejemplo hasta un objeto de gran rareza depende de la \textit{loot box} que compres. Esto hace que el jugador incremente sus expectativas e interés por esa posible mejor recompensa. De hecho cuando compras 10 cajas en el mismo mes te regalamos una caja extra, para incentivar mas aún esas compras.
        \item Niveles: los puntos de experiencia junto con las subidas de nivel ligadas a el desbloquear habilidades es un buen gancho para fomentar que los jugadores sigan jugando a Ruku. ¿Porque? Pues porque muchas veces se dice uy quiero ver como han hecho esta habilidad o tengo ganas de llegar a este nivel porque me dan esta habilidad o este objeto.
    \end{itemize}
    \item \textbf{Objetivos}: Los objetivos, misiones, logros... Llámalo como quieras la función es la misma, simplemente tienen distinto alcance en el tiempo y distinta obligatoriedad. Pero, puesto que son muy importantes, a continuación le dedicaremos un apartado entero.
    \item \textbf{Progresión visual}: En el modo aventura, que aparezcan los emblemas de los continentes en la parte inferior en forma de barra no es casual, es una barra de progresión implícita. El ponerse a hacer algo y no saber cuándo va a acabar puede ser frustrante y llevarte a abandonar. Por lo que, a la hora de implementar los objetivos (que se describirán más abajo).
    \item \textbf{\textit{Feedback} Inmediato}: Los indicadores visuales anteriores son un método también de \textit{feedback} inmediato, pero también tenemos que añadir elementos sonoros, efectos especiales; uno para que el jugador sepa que está pasando, que cuando ha pulsado en un sitio o complete algo, pueda comprobar que es así y está cargando o cualquier cosa y no es algo que no está funcionando, da más seguridad, y segundo que sea satisfactorio. Cuando juegas un juego sin color ni sonido (por lo general o si no se hace bien) es soso y no deja buen sabor de boca, pero si el jugador está contentamente recibiendo estímulos es más complicado que deje de jugar, con lo cual conseguimos nuestro objetivo una vez más de retener al jugador en nuestro juego.
    \item \textbf{Desafíos incrementales}: A parte de los niveles de dificultad de los que hemos hablado anteriormente en el capitulo de el Diseño de niveles tenemos que tener también que ajustar dentro de esa dificultad la progresión del juego, haciendo que los niveles iniciales sean más fáciles para que el jugador vaya adquiriendo las habilidades necesarias para jugar a nuestro juego. Esto se puede aplicar mediante una curva de aprendizaje añadiendo progresivamente las mecánicas y/o enemigos correspondientes y para esto en Ruku ya hemos preparado el terreno teniendo distintos continentes en los que podemos seccionar y distribuir perfectamente tanto las nuevas mecánicas como los picos de la curva de aprendizaje.
    \item \textbf{Elementos de aleatoriedad}: Estos eventos incrementan las expectativas e interés de los jugadores. Ruku es experto en esto tenemos un montón de elementos dinámicos que ofrecen diversas oportunidades, por ejemplo con las tiradas de dados nunca sabes con certeza que va a pasar o por ejemplo con los diálogos hay oportunidades únicas de obtener beneficios especiales los NPCs te pueden dar objetos o como mencionábamos en anteriores capítulos no siempre podrás reclutar a los NPC igual hay uno que solo al principio cuando hablas con el o igual tienes que ganarte su confianza. Diversos eventos dinamicos con diversas oportunidades.
\end{itemize}

\newpage
\subsection*{Objetivos}
Como se ha mencionado anteriormente, Ruku tendrá objetivos diarios y mensuales con recompensas, que corresponden a los de corto y medio plazo respectivamente, con el fin de motivar y fidelizar a nuestros jugadores.

\paragraph{Corto plazo}: Para este tipo de objetivos habrá una pantalla con un tablón de anuncios en una taberna de época medieval. El tipo de objetivos que podemos ver en este apartado pueden ser: completa 3 niveles del modo aventura, gana una partida en el modo arena, mira un anuncio para recibir la recompensa, gana un nivel con un héroe [insertar rol (tanque, healer, dps)], completa las actividades diarias.

\paragraph{Medio plazo}: La pantalla de estos objetivos será muy parecida a la anterior de hecho cuando se cambia entre pantallas será un scroll lateral. Aquí podemos ver objetivos como: juega 5 veces en el modo arena, completa los niveles más difíciles de este continente (en el mismo objetivo te dirá cuales son estos niveles y cada mes sera un continente), inflige todos los tipos de daño (te pondrá un contador y si pulsas te dirá cuales te faltan).

\paragraph{Largo plazo}: estos no tienen una fecha de caducidad como la tienen el resto de objetivos. Sin embargo tampoco te recompensan por jugarlos, estos serían equiparables a los logros. Objetivos como este pueden ser: completa el modo aventura, mata a 10 enemigos con este NPC, completa un nivel \textit{no hit} (sin que te golpeen), consigue todos los logros (o lo que es lo mismo, lo que aquí hemos llamado como objetivos a largo plazo).

\subsection*{La arena}
El hecho de que los jugadores vayan escalando posiciones cuando ganan en el modo arena y que en base a eso les den una recompensa, es perfecto. Punto uno los jugadores tienen rivalidad por estar en una liga más alta o por tener más puntos aún dentro de la misma liga, y punto dos el ofrecer una recompensa a los que queden por delante también les incentiva a jugar más. Para que las recompensas de estas ligas funcionen bien se tiene que hacer con un grupo reducido de jugadores, es decir hay que dividir las ligas en grupos reducidos, por ejemplo, de 100 o 50, ya que si hacemos la liga solo por puntos y en una de estas ligas hay 1000 jugadores por poner un número, en cuanto estás de los 800 hacia abajo ya ni lo intentas, porque piensas "No voy a conseguir pasar a todos esos viciados".