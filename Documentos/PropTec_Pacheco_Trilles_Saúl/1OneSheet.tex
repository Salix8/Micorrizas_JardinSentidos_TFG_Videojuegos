\chapter{One Sheet}
\begin{segment}
\subsection{1. Titulo del juego}
Ruku: la caída de un mundo
\subsection{2. Publico objetivo}
Adolescentes y adultos jóvenes, sobre todo amantes de los RPG y jugadores de juegos de rol y más específicamente el juego de rol Dungeons and Dragons.
\subsection{3. Resumen del juego}
Un juego RPG por turnos con proyección isométrica. En el que tienes varios personajes (entre 1-4) y luchas contra unos enemigos. Los Slaad criaturas anfibio que descienden del cielo con sus naves para intentar conquistar este planeta. La idea es que cada personaje en su turno puede usar diversas habilidades mediante una acción grande, una acción pequeña y una acción gratuita, aparte de una reacción para usar fuera de turno (lo que da más juego). Depende de la habilidad que quiera usar el personaje tendrá un coste u otro. Cada una de estas habilidades obviamente estará relacionada con el rol que adquiera cada personaje en el equipo.
\subsection{4. Distintos modos de \textit{gameplay}}
El fuerte principal del juego es un modo campaña al más estilo plantas versus zombies (enemigos a rachas en un mapa limitado con una zona que defender), y después como parte secundaria del juego habrá un modo online en el que poner a prueba la combinación de tus personajes contra las de otro jugador y así ascender en el ranking global y codearse con los mejores jugadores.
\subsection{5. ¿Porque este juego es original?}
Es un juego que implementa un mundo de fantasía medieval con una invasión de Slaads, juntando así la magia de este mundo fantastico medieval contra la tecnología traída desde otra parte del multiverso. Esta fusión de elementos crea un entorno único y fascinante que no se encuentra comúnmente en otros RPGs.
\vspace{1cm}
\subsection{6. ¿Porque este juego es interesante?}
Narrativa envolvente: La historia de un mundo en caída libre debido a una invasión tecnológica, en la cual el jugador es responsable de escasa resistencia que queda en este mundo todo esto sin contar que solo hay unos pocos usuarios capaces de lanzar magia con la que hacer frente a la alta tecnología con la que se les asedia. No obstante aun quedan usuarios diestros en las artes marciales que rivalizan con los usuarios de magia.

Estrategia profunda: La combinación de acciones grandes, pequeñas y gratuitas, junto con reacciones fuera de turno, ofrece una jugabilidad rica y estratégica. A todo esto hay que añadirle las diferentes clases y habilidades de los personajes que puedes juntar en tu equipo.

\subsection{7. ¿Cuales son las principales diferencias con los juegos del mismo genero?}
Sistema de progresión de mundo dinámico: A diferencia de juegos como Divinity: Original Sin y Wasteland 3, donde el mundo es estático, en Ruku: la caída de un mundo, las decisiones del jugador afectan directamente el entorno y la narrativa, cambiando la disposición de los enemigos, aliados y recursos disponibles.

Economía y gestión de recursos: A diferencia de Baldur’s Gate 3, Ruku incorpora un sistema de economía y gestión de recursos donde los jugadores deben recolectar y gestionar materiales para mejorar su equipo y habilidades, lo que añade una dimensión estratégica adicional.

Modo cooperativo asimétrico: A diferencia de muchos RPGs tradicionales, Ruku ofrece un modo cooperativo asimétrico donde un jugador puede controlar a los héroes y otro a los enemigos, creando una experiencia competitiva y colaborativa única.
\end{segment}

