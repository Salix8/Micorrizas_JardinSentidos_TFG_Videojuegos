\chapter{Reglas}
\begin{segment}
\paragraph{Objetivo:}
El objetivo principal de Ruku es el modo campaña en el cual debes defender los refugios de la resistencia del ataque de los Slaad, usando los diferentes tipos de personajes con sus habilidades y objetos. Además cada refugio tendrá unas características distintas y esto sumado a que a medida que avanzas en los niveles, el jugador se va a enfrentar a Slaads más fuertes con nuevas habilidades y patrones, obliga al jugador a ajustar constantemente su estrategia para salir victorioso y así liberar a Ruku de la invasión.
Como se acaba de comentar en la taxonomía de Bartle no todo va a ser enfocado al modo campaña con la \textit{main history}. El juego también cuenta con diferentes niveles a modo de retos (muy difíciles modo estratega, para los jugadores más \textit{hardcores}) y un modo infinito para cada continente en los que el jugador pueda ponerse a prueba y ver cuanto tiempo aguanta contra las oleadas de los Slaads. 
Por ultimo no podemos dejar de lado el modo multijugador en el que ganas si eres mejor que tu oponente, ya sea porque has obtenido más puntos o porque este ha muerto.
\end{segment}


%---------------------------------------------
\paragraph{Limites:} restricciones a tener en cuenta.


\begin{enumerate}
    \item \textbf{Espaciales}: El mapa es una cuadricula cual tablero de ajedrez de tamaño variable dependiendo de nivel. En cada casilla solo se puede situar una criatura como norma general.
    
    \item \textbf{Turnos}: El juego va por rondas y cada ronda tiene el turno de todos los personajes en el campo de batalla. Por lo general cada personaje solo puede actuar en su turno (exceptuando una pequeña acción situacional que veremos más adelante que se llama reacción).
    Como extra en algunos de los niveles el jugador tendrá un numero reducido de turnos en los que superar los obstáculos (por lo general alguna oleada de Slaads).

    
    \item \textbf{Economía de acción (Recursos)}: Hablando de turnos que pueden hacer los personajes en uno de estos turnos. Todos los personajes tienen dos acciones, una más grande con la que se pueden hacer cosas que cuestan más, como por ejemplo hacer un ataque o lanzar un conjuro y una más pequeña con la que se podrán hacer cosas cosas menos potentes, por ejemplo activar una habilidad o curarte un poco. A estas dos acciones las llamaremos \underline{Acción} y \underline{Acción adicional} respectivamente (cada personaje solo dispondrá de una de cada por turno).
    
    Además de estas dos acciones hay cosas que no tendrán ningún coste por ejemplo pasar un objeto de un personaje a otro, eso sera gratis. Como parte de estos elementos que no cuestan nada también tenemos el \underline{Movimiento}, sin embargo este es un consumible por si mismo dependiendo del personaje solo te puedes mover x casillas cada turno, aunque lo puedes hacer cuando quieras (cuidado con dar la espalda a tus enemigos). 
    
    Y por ultimo están las \underline{Reacciones} que son lo que puedes hacer fuera de tu turno en respuesta a algo especifico que ocurra, por ejemplo cuando vayan a pegar a un aliado que este en una casilla adyacente interpones el escudo entre el ataque y tu aliado.
\end{enumerate}

%---------------------------------------------
\section*{Jugadores}
Características y descripción de los jugadores en Ruku.
\begin{enumerate}
    \item \textbf{Tipos de Jugadores}
    \begin{itemize}
        \item \textbf{Solitario (Solo)}: Este es el principal publico objetivo en realidad la mayor parte del contenido esta creado para jugar de este modo, o bien en el modo campaña jugando los niveles en el que el jugador controla de 1 a 4 personajes y defiende los refugios de los Slaads o bien en los niveles de desafíos.
        \item \textbf{Competitivo}: En el modo multijugador, los jugadores compiten entre sí, poniendo a prueba la combinación de personajes y estrategias (incluyendo aquí la compra de debuffs) para ascender en el ranking global.
    \end{itemize}
    
    \item \textbf{Roles de los Jugadores}
    \begin{itemize}
        \item \textbf{Combatiente}: Jugadores que se enfocan en maximizar el daño y eliminar a los enemigos rápidamente asumiendo ciertos riesgos que de otra forma no son comunes.
        \item \textbf{Estratega}: Jugadores que prefieren planificar sus movimientos y utilizar las habilidades de sus personajes de manera táctica, para así minimizar el uso de estas.
        \item \textbf{Explorador}: Jugadores que disfrutan descubriendo niveles secretos y explorando las diferentes zonas del juego.
        \item \textbf{Soporte}: Jugadores más defensivos que se centran en curar y proteger a sus aliados, para aumentar la supervivencia de la \textit{party}.
    \end{itemize}
    
    \item \textbf{Interacción entre Jugadores}
    \begin{itemize}
        \item \textbf{Cooperación}: Los jugadores pueden copiar la \textit{build} del personaje de un amigo (si tiene todos los elementos necesarios), lo que fomenta que se agreguen a amigos los unos a los otros.
        \item \textbf{Competencia}: En el modo multijugador, los jugadores compiten para demostrar a cualquier precio quién tiene la mejor puntuación en el ranking global. Además también pueden competir para ver quien tiene más logros o insignias.
        \item \textbf{Comunicación}: Los jugadores pueden comunicarse a través de \textit{stikers} dentro del modo multijugador.
    \end{itemize}
    
    \item \textbf{Jugador vs. Entorno (PvE)}
    \begin{itemize}
        \item El formato principal de este juego el modo campaña, donde los jugadores deben defender los refugios de la resistencia de las hordas de los enemigos (Slaads) controlados por IA y completar objetivos, para avanzar en los diferentes niveles y acabar con esta invasión.
    \end{itemize}
    
    \item \textbf{Personalización de Jugadores}
    \begin{itemize}
        \item \textbf{Progresión}: A medida que avanzan en el juego, los personajes van subiendo de nivel ganando así nuevas habilidades. Ademas hay niveles que cuando los acabas encuentras objetos útiles tanto para los personajes como para comerciar y comprar cada vez mejores armas. 
        \item \textbf{Roles}: Como es un RPG, cada personaje tiene un rol asociado con su clase y/o subclases, es decir se adapta a la forma de jugar del jugador, si le gusta más jugar un personaje a \textit{mele} elegirá un personaje con una clase con habilidades afines a ese tipo de juego. Esto si sumamos la subclase que es como una especialización de lo que se ha elegido anteriormente, el personaje que tiene cada jugador es bastante a su propio gusto podríamos decir.
        \item \textbf{Avatares}: Los jugadores pueden personalizar la apariencia de sus personajes con diferentes \textit{skins}, accesorios y como no los objetos. 
        
        Los objetos son importantes porque pueden proporcionar a tu personaje habilidades útiles para los diferentes retos. Aquí es donde los estrategas miraran que los objetos te den las mejores habilidades y los jugadores que no sean tan \textit{hardcore} igual le ponen un objeto que se vea más bonito estéticamente en su personaje. Otro punto a tener en cuenta es que al ser los objetos los que te dan ciertas habilidades abre más el abanico de personalización de los personajes, porque es más difícil que personajes de distintos jugadores se parezcan entre si (si el juego esta balanceado).
    \end{itemize}
    
    \item \textbf{Comunidad de Jugadores}
    \begin{itemize}
        \item \textbf{Eventos}: Ruku puede incluir eventos especiales o de temporada que ofrezcan desafíos únicos o recompensas exclusivas.
        \item \textbf{Plataformas de Comunidad}\label{sec:plataformas_comunidad}: Los jugadores pueden compartir estrategias, \textit{builds}, experiencias y contenido relacionado con el juego en foros, redes sociales e incluso crear una pagina de \href{https://www.fandom.com/topics/games}{Fandom} en la que recopilar toda la información como en muchos videojuegos y otros medios de ocio.
    \end{itemize}
\end{enumerate}


%---------------------------------------------
\section*{Obstáculos y Conflictos}
\begin{enumerate}
    \item \textbf{Entornos}
    \begin{itemize}
        \item \textbf{Terreno Difícil}: No todo el terreno es igual y no siempre se lucha en las mejores condiciones, ocasionalmente los personajes defenderán zonas antes de que los Slaads lleguen a los refugios y se pueden encontrar con pantanos o terrenos muy frondosos pasar por casillas con este tipo de terreno ralentizan nuestro movimiento a la mitad.
        \item \textbf{Clima Adverso}: Puede haber niveles con condiciones climáticas que afecten la visibilidad y el rendimiento de los personajes.
    \end{itemize}
    \item \textbf{Agentes del juego}
    \begin{itemize}
        \item \textbf{Interacciones con los NPCs}: Puedes que no hayas tomado las decisiones que beneficien a todo el mundo o hayas dejado de lado a cierto personaje, esto repercutirá directamente en la relación que tengáis y puede que a partir de ahí te use solo como carnada o distracción para sus enemigos esto se reflejara en el tipo de misiones que te da o incluso puede que deje de hablarte.
        \item \textbf{Traición y Alianzas}: Hablando de las relaciones con los NPCs no todos los NPCs que te encuentres tienen porque ser legales y buenos, puede que haya personajes desconfiados otros muy preocupados de hacer buenas relaciones y alianzas pero otros que te traicionen a la primera de cambio si el cree que puede prolongar la supervivencia de su comunidad.
        \item \textbf{Slaad}: Criaturas hostiles que atacan en grupos y presentan diferentes habilidades y resistencias, obligando a los jugadores a adaptar sus estrategias. Usualmente equipados con tecnología avanzada que solo ellos pueden usar.
        \item \textbf{Jefes de Zona}: Enemigos poderosos que actúan como un desafiante atacante o ya de cara al final del juego, cuando el jugador empiece a repeler la invasión, como guardianes de áreas clave, ofreciendo un reto significativo y valiosas recompensas al ser derrotados.
    \end{itemize}
    \item \textbf{Conflictos Internos (de la \textit{party})}
    \begin{itemize}
        \item \textbf{Dilemas Morales}: Dependiendo de las decisiones que tome el jugador al igual que en el primer punto del apartado anterior, si tomas alguna decisión que algún personaje que llevas en la \textit{party} no esta de acuerdo se te hará saber y si sigues por ese camino es probable que te acaben abandonando.
        \item \textbf{Gestión de Recursos}: El jugador debe equilibrar el uso de los recursos limitados, como por ejemplo las pociones, pergaminos y lo más importante los objetos, saber a quien le pones que objeto puede ser clave para la supervivencia de tu \textit{patry}.
    \end{itemize}
    \item \textbf{Puzzles y Trampas}
    \begin{itemize}
        \item \textbf{Rompecabezas Ambientales}: En medio de cualquier nivel de cada continente o incluso esparcido entre estos (niveles) hay algún elemento para poder resolver un acertijo y así desbloquear una área secreta, donde conseguir objetos especiales. 
        Y ahora por otro lado dando un giro de 180º hay otro tipo de Rompecabezas ambientales los \textit{bosses}. No todos los \textit{bosses} van a ser una bolsa de vida a veces hace falta pensar un poco por lo que 
        \item \textbf{Trampas Ocultas}: Los Slaads no nos van a poner las cosas fáciles, pueden usar alguna de sus acciones para poner trampas que inflijan daño o algún que otro \textit{debuff
}. Y a medida que os acerquéis a sus naves estarán más camufladas con su tecnología. Aunque si tenéis un personaje suficientemente ducho y las detecta antes de que se activen, siempre se pueden intentar desactivar.
    \end{itemize}
\end{enumerate}


\section*{Reglas}

%---------------------------------------------
\paragraph{Reglas Fundacionales:} 
\begin{enumerate}
    \item \textbf{Mecánicas Básicas}
    \begin{itemize}
        \item \textbf{Turno}: Todos los niveles están divididos en rondas, estas rondas, albergan el turno de todos los personajes en un orden determinado, al inicio de cada ronda se asigna una iniciativa (para ver quien tiene el turno antes y quien después), esto se hace en base a los \textit{stats} de cada criatura que haya en el campo de batalla.
        \item \textbf{Movimiento}: A lo largo del turno de cada personaje puede gastar su velocidad de movimiento cuando lo desee, antes o después del resto de sus acciones. Como norma general todos los personajes se moverán 6 casillas. Y como parte del movimiento un personaje puede caminar, es decir moverse en cualquier dirección hasta un máximo de su velocidad de movimiento (6 casillas). Saltar o trepar te cuesta 1 punto de movimiento por cada casilla que quieras recorrer usando este movimiento. Y Nadar en el agua es terreno difícil por lo que te cuesta el doble de movimiento.
        \item \textbf{Economía de acción}: Todos los personajes tienen habilidades que pueden usar a lo largo de su turno, desde las más simples como atacar con una espada, hasta las más enrevesadas como lanzar algún tipo de conjuro complejo. Para esto se utilizan las Acciones y Acciones adicionales, en cada turno tienes una de cada (al igual que reacciones que hablaremos más adelante). Puedes usar tu Acción para realizar un ataque con arma, lanzar un conjuro, esquivar, usar un objeto, empujar, esconderse, correr, desengancharse. Y la Acción adicional para usar algún rasgo de tu clase, lanzar algún conjuro o realizar un ataque con el arma secundaria.
        La reacción se utiliza en algún conjuro pero sobretodo para los ataques de oportunidad, es decir cuando te vas del rango de acción de una criatura (les das la espalda), pues ellos pueden aprovechar para hacerte un ataque con el arma que estén empuñando.
        \begin{itemize}
            \item \textbf{Usar objetos}: Hay objetos muy variados y estos pueden otorgar al personaje alguna pasiva, es decir, alguna habilidad que esta constantemente activada sin que el personaje tenga que hacer nada. También puede otorgarte una habilidad que necesites usar la Acción para gastar la habilidad incluso alguno de sus usos.
            \item \textbf{Ataques}: Puedes usar tu acción para realizar un ataque con armas contra otra criatura, esta acción tiene una probabilidad de acertar o no acertar esta se determina en función de la defensa del la criatura atacada. Para hacer esto usaremos el estilo más rolero, se tirara en dado de 20 en la parte de la derecha de la pantalla y en función de lo que salga sumaremos el \textit{stat} con el que estemos atacando y si es mayor o igual a la defensa del enemigo le habremos dado y con ello recibirá el daño u efecto de nuestro ataque.
            \item \textbf{Conjuros}: Hay dos tipos de conjuro, los que lanza el \textit{caster} y los que tiene que esquivar el resto de criaturas. Si lo lanza el \textit{caster}, funciona muy parecido a un ataque con armas solo que el \textit{stat} que se sumará al dado de 20, será con el que lanzas la magia, un ejemplo de esto seria te disparo un virote de fuego. Ahora si son las otras criaturas las que tienen que eludirlo entonces estas hacen una tirada de resistencia a la que sumaran también un \textit{stat} depende de lo que haya que evadir, por ejemplo un \textit{caster} crea una nube de gas pues las criaturas dentro de esta nube tiene que ver si son lo suficientemente fuertes o tienen la suficiente constitución como para resistir esto. Todas estas tiradas al igual que en los ataques se realizaran con d20. Y el con que \textit{stat} se resiste cada conjuro se especificara en la ficha del conjuro en su descripción.
        \end{itemize}
    \end{itemize}
    \item \textbf{Mecánicas más profundas}
        \begin{itemize}
            \item \textbf{Stats}: Hemos hablado mucho de los \textit{Stats} pero como funcionan en este juego. Bien pues aquí cada personaje tiene 6 \textit{stats} base, luego puede tener habilidades como a la hora de obtener materiales obtiene el doble o tiene don de gentes pero eso ya se explicara como funciona en cada clase. Los  \textit{stats} son:
            \begin{itemize}
                \item Vida (HP): La vida de las criaturas indica la cantidad de daño que soportan antes de morir. Cuando mueres obviamente no puedes hacer nada. (La abreviación HP esta en ingles porque esta más estandarizado en este nicho).
                \item Clase de armadura (CA): Este numero indica la dificultad de acertar el golpe la magia o lo que sea. Como se explica en la acción de ataque se tira un d20 y se suma lo que se tenga que sumar por el arma o lo que use esa criatura, y si el numero obtenido es < que la CA entonces o acierta, si el numero obtenido es >= que la CA entonces si que aciertas y le haces el daño o efecto que corresponda.
                \item Bono de competencia (BC): es un modificador que se utiliza para añadirlo en algunos sitios para equilibrar el sistema. Por ejemplo, se añade a la tirada de dificultad para ver si le aciertas el golpe a un enemigo no. Este bonificador empieza en +2 y se incrementara e uno cuando se complete un continente, como hay 5 continentes se llegara al +6, aunque este solo se podrá aprovechar en el \textit{postgame}.
                \item Fuerza (Fue): con lo que se tirarán todos los ataques con armas a \textit{melé}, y se resistirá efectos como empujes.
                \item Destreza (Dex): se tiran todos los ataques con armas a distancia y se resisten efectos que tengas que esquivar (típico: cae una bola al más estilo \textit{Indiana Jones}).
                \item Constitución (Con): sirve para calcular la vida y para resistir venenos.
                \item Inteligencia (Int): Uno de los atributos con los que se puede lanzar magia, este atributo también va ligado a la memoria y la lógica.
                \item Sabiduría (Sab): Otro atributo con el que poder lanzar magia, además este también sirve para darte cuenta de las cosas que te rodean.
                \item Carisma (Car): El ultimo atributo con el que también puedes lanzar magia, y también influye en las relaciones con el resto de criaturas.
            \end{itemize}

            
            \item \textbf{Salud}: Cada personaje tiene una vida determinada que se calcula en función de su clase, los \textit{stats} y el nivel del personaje. Dependiendo de en que nivel (del mundo) que quiera jugar el jugador los personajes estarán a un nivel u otro.
        \end{itemize}
    \item \textbf{Entorno}
    \begin{itemize}
        \item \textbf{Daño}: Hay muchas situaciones que pueden dañar a los personajes tanto los del jugador como los personajes enemigos, y los Slaad no dudaran en aprovechar el terreno para acabar con los personajes del jugador. Por ejemplo si empujan a alguien y este cae desde una altura determinada se hace daño, cada 2 casillas que te caigas recibirás 1d6 de daño contundente. Otro ejemplo seria el terreno difícil de que se ha hablado antes a veces solo nos costara movimiento, pero otras en el caso de que sea lava u otros elementos que ese terreno haga daño. Un derivado de este sea que haya agua cerca y por lo que sea, ya sea la tecnología de los Slaad o electricidad de alguna fuente mágica toque el agua lo que haría otro d6 de daño de rayo.
        \vspace{0.1cm}
        \item \textbf{Zonas Seguras}: En todos los continentes hay un nivel en el que estas en las puertas del refugio defendiéndolo, esto implica que de una forma u otra hay algunas posiciones dentro de este mapa que son bastante más seguras, estas se caracterizaran porque el enemigo lo tendrá más difícil para darte o porque tendrás algún tipo de \textit{bufo} para poder superar con comodidad a esos enemigos, obviamente sin dejar de lado que tu \textit{party} es la importante y que es gracias a esta por lo que consiguen ganarle terreno al enemigo, con esto buscamos hacer sentir al jugador que es parte de la narrativa que es importante para la historia.
        Por otro lado "las zonas seguras" en un videojuego se entienden como los lugares tranquilos, zonas donde no tengo que andar preocupado por si me van a emboscar atacar o cualquier derivado de estos. En este juego esas zonas no existen como tal ya que es un juego de niveles, esto quiere decir que al estar jugando al juego "la zona segura" seria las pantallas en la que crear tu \textit{party}, en las que eliges el nivel, consultas el ranking global, tablero de misiones, etc.
        \item \textbf{Materiales}: Los jugadores pueden designar a personajes con los que no quieran jugar durante x batallas para que vayan a recolectar ciertos materiales (\textit{forrajear}), estos pueden servir para la crear ciertos objetos como pociones o trampas incluso materiales para hacer mejoras en algún tipo de equipamiento no mágico.
        \item \textbf{Inventario}: Como acabamos de ver en el apartado anterior los personajes pueden conseguir ciertos materiales pero no pueden cargar infinitos materiales, por lo que según los \textit{stats} de los personajes podrán cargar más o menos materiales (u objetos). Por lo que tendremos que tenerlo en cuenta cuando enviemos a un personaje a \textit{forrajear}.
        \item \textbf{Destructibilidad}: Dentro del mapa hay muchas partes de este que pueden ser destruidas y tienen sus consecuencias, si hay una plataforma de madera por ejemplo y alguien le prende fuego la plataforma turno a turno comenzara a arder y desmoronarse lo que provocara que si hay alguien ahí caiga a la zona de abajo. También puede que si al jugador le da por destruir una pared aparezca una mina que igual es el camino para descubrir un nivel secreto u os da la ventaja en la siguiente nivel al sorprender a los enemigos.
    \end{itemize}
    \item \textbf{Modos de Juego}
    \begin{itemize}
    \item \textbf{Estándar}: Dentro de esta modalidad están los dos modos de juego habituales, es decir los que están siempre disponibles. Por un lado tenemos el modo solitario que es el juego en modo aventura por niveles conociendo la trama de la historia y es un modo \textit{offline}. Y por otro lado en el modo \textit{online} tenemos el modo de juego multijugador que es un modo de juego más competitivo.
    \item \textbf{Eventos}: La idea es crear eventos con nuevas misiones y recompensas como incentivo y para aumentar el indice de participación en el juego, estos eventos se realizaran en base a festividades como bien pueden ser las navidades \textit{halloween}.
    \end{itemize}
\end{enumerate}

%---------------------------------------------
\paragraph{Reglas Operacionales:} 
\begin{enumerate}
    \item \textbf{Primera partida}
    \begin
{itemize}
        \item \textbf{Creación de personaje}: Nada más empezar el juego el jugador tiene que crearse un personaje. Esto brinda al jugador una personalización increíble ya que opta a una gran variedad de opciones tanto en estilísticas como funcionales. Estas pasan de elegir colores y formas de cuerpos, cabezas, ojos, cuernos entre otros complementos y por la parte de las opciones más funcionales tenemos la elección de la raza y clase (la subclase vendrá más adelante cuando se ganan un par de niveles).
        
        \item \textbf{Tutorial}: Aquí aparecerá el mentor (solo en cuadro de dialogo no veremos nunca su personaje en el campo de batalla), este nos indicara las cosas básicas a modo de tutorial de hecho hasta nos proporcionara algún que otro apunte que tiene por su cuaderno a medida que avancemos nos ira desbloqueando cada vez más información (así no abrumamos al recién iniciado con demasiada información). A partir de aquí empezaremos en el primer continente el cual tendremos que liberar de la invasión Slaad. Esto lo podremos hacer solos o acompañados de otros personajes depende de lo que quiera el jugador y los personajes (habrá que tirar de carisma si queremos convencerles de que se unan a nuestra épica aventura llena de retos).
    \end{itemize}
    \item \textbf{Nivel}
    \begin{itemize}
        \item \textbf{Antes de empezar}: El jugador aquí vera el mapa, lo cual ya es una ventaja, ya que con ello puede ver la posición de todos los personajes y todas las zonas interesantes (barriles explosivos etc) en el campo de batalla. Pero antes de que empiecen las rondas (el conjunto de turnos de todas las criaturas en el campo de batalla), se tiene que calcular la iniciativa de cada uno para determinar quien va antes y quien después esto se calcula con un porcentaje al que le añadiremos el \textit{stat} de la destreza. Una vez tengamos el orden de iniciativa empieza la ronda con el turno de cada uno. Cuando acabe la ronda se reinicia este proceso, es decir se vuelve a calcular el orden de iniciativa y empieza otra vez la ronda y así todo el rato hasta que salte la \textit{win condition} que se acabara el nivel.
        
        \item \textbf{Turno}: Ya se ha comentado en otros apartado pero aqui lo vamos a resumir. Tienes movimiento, acción y acción adicional, para usar cuando quieras dentro de tu turno. Para saber que gasta que, lo especificara en cada habilidad pero para mayor claridad hay unos símbolos al lado de la zona donde eliges que habilidad gastar que tendrán un color, cuando selecciones una habilidad se iluminara lo que vas a gastar y cuando la hayas gastado este se quitara dejando solo la forma del símbolo (para el movimiento una linea que se vaya consumiendo dependiendo de las casillas que le queden al personaje, la acción un circulo verde y la acción adicional un triangulo naranja).
        
        \item \textbf{\textit{Win condition}}: Cada nivel conlleva un reto u objetivo no es un mapa con un combate y ya esta. El reto viene cuando hay que proteger algo, actuar en un sitio donde no funciona la magia, donde hay una nave nodriza. Todo son variaciones con las mecánicas del combate. Esto proporciona que los niveles sean suficientemente diferentes para que al jugador no le parezca monótono incluso le parezca divertido el decir "uf ahora no puedo hacer que los enemigos pasen de aquí". Por ejemplo el primer nivel de cada continente, que como bien se ha mencionado anteriormente iba a ser en el refugio de dicho continente y en este los jugadores iban a tener una clara ventaja respecto a los Slaads ya sea por las zonas seguras o las ayudas que se puedan recibir de estos refugios, volviendo al tema, aquí el objetivo esta más que claro, es ayudar a salvar el refugio de los Slaads.
        
        \item \textbf{Recompensa}: Al acabar muchos niveles tendremos algún tipo de recompensa o bien oro por parte de los personajes para después comerciar con otros personajes o bien algún tipo de material para crear pociones, u objetos que puedan usar nuestros personajes.
    \end{itemize}
    \item \textbf{Fuera del nivel}
    \begin{itemize}
    \item \textbf{Misiones}: A parte de las misiones que te puedan dar los personajes dentro de los niveles, también hay misiones fuera de estos. Estas misiones se clasifican en diarias semanales y mensuales y otorgan unas recompensas proporcionales al tiempo que haya que dedicarles. Las recompensas luego se pueden utilizar para comprar objetos.
    \end{itemize}
\end{enumerate}

%---------------------------------------------
\paragraph{Reglas Escritas:} 
\begin{enumerate}
    \item \textbf{Tutorial}: son unas instrucciones básicas guiadas paso a paso, en las cuales te dejan probar en un entorno ficticio como se van a desarrollar los retos, reglas y mecánicas durante el juego. 

    \item \textbf{Libros}: Para los más fanáticos del genero en su versión de mesa, el juego cuenta con una pantalla que tiene un formato a modo de libro de reglas de juego de rol, aunque cuenta con un buscador de palabras para que no se haga tedioso el ir buscando las mecánicas pagina por pagina. Sin embargo si tu no eres de esos hay un botón en el que desactivar este formato, esto implica que ahora veras todas las mecánicas y reglas escritas en un indice con desplegables.
    
    \item \textbf{Guias}: Como bien se ha mencionado antes en las Plataformas de la comunidad en el punto \ref{sec:plataformas_comunidad} de los jugadores , habra muchas cosas subidas a la pagina de \href{https://www.fandom.com/topics/games}{Fandom} para que los jugadores consulten información del juego. 

\end{enumerate}

%---------------------------------------------
\paragraph{Reglas de Comportamiento:} 
\begin{enumerate}
    \item \textbf{Interacción}: Las interacciones básicas como se ha mencionado ya en diversos apartados son: movimiento, acción, acción adicional y reacción. Por supuesto estas luego se pueden desglosar en muchísimas más opciones que tendrá que ir descubriendo el jugador y que dependerán de lo que este haya escogido. Pero cosas que tengan en común todos los personajes son por ejemplo en el movimiento: correr, saltar, nadar, trepar... , con la acción: realizar un ataque, lanzar un conjuro (depende de la clase), usar objetos... , acción adicional el ataque con la mano mala, rasgos de clase...
    
    \item \textbf{Conducta}: En cuanto a la conducta de los propios jugadores se estima que pondrán de su lado para fomentar un ambiente de respeto entre los jugadores. En Ruku para evitar lenguaje ofensivo, insultos o cualquier forma de acoso, el juego solo da la opción de enviar \textit{stikers} y frases ya preconstruidas. Además el uso de trampas, \textit{hacks} o cualquier método que dé una ventaja injusta a los jugadores esta totalmente prohibido por lo que a estos jugadores se les \textit{baneara} la cuenta de forma permanente.
    
\end{enumerate}



%---------------------------------------------
%\section*{Competición}
\begin{segment}[Competición]
    \begin{enumerate}
        \item \textbf{PvP}: En este modo de juego, los jugadores luhan unos contra otros en igualdad de condiciones, cada uno usa el equipo que ellos eligen 4 personajes que entrar en un campo de batalla. En este campo de batalla se tienen que enfrentar a una oleada de enemigos ya sean los Slaads como alguna de sus variaciones o aliados. Adicionalmente el oponente puede gastar sus monedas que gana matando a estos enemigos, para invocar enemigos en tu campo de batalla o que no puedas usar alguna habilidad (es decir son monedas que solo se adquieren y se pueden gastar en esta partida multijugador).
        \vspace{0.4cm}
        \item \textbf{Ranking global}:  Los jugadores en el apartado anterior recibirán una puntuación en función de a cuantos jugadores hayan derrotado y si les han ganado de mucho no (aquí se debería tener en cuenta si han ganado por abandono perdida de internet etc). El problema es que no es lo mismo ganar de goleada que ganar por los pelos, por lo que en Ruku se quiere premiar a quien ha conseguido ganar de sobra, por ello se le darán más puntos y subirán antes en el ranking este tipo de jugadores. 
    \end{enumerate}
\end{segment}

%---------------------------------------------
\section{Consejos}
\begin{segment}
    \begin{enumerate}
        \item \textbf{Consejos de inicio}: Este tipo de consejo aparecerán en las pantallas de carga que hay al entrar en un nivel y al salir.
        
        \item \textbf{\textit{Builds}}: Estos consejos vendrán dados de la mano del diario de nuestro mentor el cual nos ira "desbloqueando" hojas en las que el tiene estrictas y dibujadas ciertas \textit{builds} con combos específicos que se dan al elegir ciertas clases con ciertas habilidades con ciertos objetos. El diario se podrá ver en la pantalla de creación de personajes y tendrá anotaciones y apuntes en sucio como si fuese una libreta de un diario antiguo de verdad.
        
        \item \textbf{Estrategias de juego}: En las pantallas de carga entre la selección de niveles y entrar al nivel, en la parte inferior arriba de la barra de carga aparecerán consejos estratégicos. Por ejemplo con "si disparas con tecnología, fuego o rayo a un barril explosivo explotará, cuidado por los alrededores".

        \item \textbf{Recomendaciones \textit{in game}}: Una vez en el campo de batalla cuando a un personaje le quedan cosas por hacer/ por gastar, en un azul translucido, aparecerá parpadeando los movimientos o acciones que el personaje puede hacer, de aqui solo aparecerá una recomendada, al igual que en muchos juegos de ajedrez, esto no quiere decir que sea la mejor jugada o movimiento. Pero es importante ya que te marca que te quedan consumibles que o bien porque no quieres o bien porque no te has acordado no los has gastado.
        
    \end{enumerate}
\end{segment}