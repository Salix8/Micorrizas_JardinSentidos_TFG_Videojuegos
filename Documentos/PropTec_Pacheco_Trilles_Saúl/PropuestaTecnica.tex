\subsection{Abstract}
This document presents the technical proposal for a Final Degree Project in the Degree in Design and Development of Video Games. The project aims to develop Micorrizas: Nature's Lost Legacy, an educational video game whose main objective is to promote knowledge of and engagement with the natural environment through active exploration and collaborative learning.

The design of the game is based on a real pedagogical activity carried out with university students of Early Childhood Education in the subject MI1815 - The Natural Environment in Early Childhood Education, which consists of a guided visit to the “Jardín de los Sentidos”, an outdoor space at Jaume I University characterised by high biodiversity. This project is based on the transformation of that activity into a playful and interactive experience, using game design principles to encourage observation, discussion and reflection, while avoiding the mere digitalization of theoretical content.

The game will be a cooperative, location-based mobile game developed with Unity, incorporating geolocation and a lightweight multiplayer system based on a local host and cloud synchronization to support group decision-making and shared progress.\\



\subsection{Keywords}
Serious game - Mobile games - Cooperative gameplay - Experiential learning - Environmental education



\subsection{Introduction and Motivation of the Work}
The project consists of the design and development of ‘Micorrizas’, a cooperative educational video game for mobile devices, conceived as a serious game based on geolocation.

The game is designed to be played on site in the “Jardín de los Sentidos”. The main objective of the project is to promote knowledge of the natural environment and the ability to use it as an educational resource. To this end, the design of the video game is based on a real educational activity taught in the MI1815 - The Natural Environment in Early Childhood Education course of the Early Childhood Education Degree at the Jaume I University (UJI).

Currently, the activity is carried out through traditional guided tours, which have certain limitations in terms of active student participation, as well as logistical constraints such as weather and schedules, and the availability of a trained instructor to lead the activity. Rather than simply digitising theoretical content, the video game uses narrative and mechanics based on exploration, observation and group decision-making to encourage active debate and reflection. This approach allows students to identify natural elements and understand how the environment can be used as a teaching resource.

Technically, the game will be developed using Unity for mobile devices and will feature a lightweight multiplayer system based on a local host and cloud synchronisation.



\subsection{Related Subjects}
\begin{itemize}
    \item \href{https://ujiapps.uji.es/sia/rest/publicacion/2025/estudio/231/asignatura/VJ1238}{VJ1238 - Fundamentals for the Design of Educational Video Games}
    \item \href{https://ujiapps.uji.es/sia/rest/publicacion/2025/estudio/231/asignatura/VJ1220}{VJ1220 - Databases}
    \item \href{https://ujiapps.uji.es/sia/rest/publicacion/2025/estudio/231/asignatura/VJ1222}{VJ1222 - Video Game Concept Design}
    \item \href{https://ujiapps.uji.es/sia/rest/publicacion/2025/estudio/231/asignatura/VJ1208}{VJ1208 - Programming II}
    \item \href{https://ujiapps.uji.es/sia/rest/publicacion/2025/estudio/231/asignatura/VJ1209}{VJ2309 - 2D Design}
\end{itemize}



\subsection{Objectives of the TFG}
\begin{itemize}
    \item To design a cooperative educational video game that promotes active learning through exploration and observation of a real environment.
    \item To develop a mobile game that reacts to the player's physical position within the Garden of the senses.
    \item Offer teachers of the subject in collaboration a valid and entertaining educational tool for independent activity.
    \item To provide an educational and entertaining tool to raise awareness of the environment and its biodiversity.
\end{itemize}



\subsection{Task and Time Planning}
\begin{tablebox}{Task and Time Planning}
\begin{tabulary}{\linewidth}{c L L}
    \textbf{Task} & \textbf{Description} & \textbf{Estimated Hours} \\
    \hline
    Project analysis & Study of the educational context and requirements & 10 \\
    Game design & Definition of mechanics, narrative and gameplay loop & 50 \\
    Technical design & Architecture and system planning & 30 \\
    Art design & Design and draw assets & 10 \\
    Development & Implementation in Unity & 100 \\
    Multiplayer \& geolocation & Integration of host-client and cloud sync & 40 \\
    Testing and iteration & Debugging and design refinement & 30 \\
    Documentation & Writing the final report & 20 \\
    Presentation & Preparation of oral defense and materials & 10 \\
    Total &  & 300 \\
\end{tabulary}
\end{tablebox}


\subsection{Expected Results}
At the end of the project, the following results are expected:
\begin{itemize}
    \item A functional prototype of a mobile educational video game to teach about the biodiversity of the sensory garden.
    \item A cooperative gameplay experience playable by small groups of students in a real outdoor environment.
    \item A system that integrates geolocation and shared game state synchronization.
    \item A documented evaluation of the system through quantitative/qualitative assessment by user testing
\end{itemize}



\subsection{Tools}
\begin{tablebox}{Tools}
\begin{tabulary}{\linewidth}{c L}
    \textbf{Tool} & \textbf{Description} \\
    \hline
    Game engine & Unity (C\#) and Visual Studio \\
    Cloud data synchronization & backend ligero (Firebase/Supabase) \\
    APIs y SDKs & Mapbox SDK (para mapas y geolocalización) \\
    Art & Clip Studio \\
    Version control & GitHub \\
    Task and Time Planning & Excel and Trello \\
    Documentation & Latex \\
\end{tabulary}
\end{tablebox}



\subsection{References}
Pitarch García, R. (2012). Guia de la flora Ornamental de la Universitat Jaume I : Un campus per a la biodiversitat. Universitat Jaume I. Servei de Comunicació i Publicacions. \href{https://lectura-unebook-es.eu1.proxy.openathens.net/viewer/9788480219457/2}{https://lectura-unebook-es.eu1.proxy.openathens.net/viewer/9788480219457/2}

