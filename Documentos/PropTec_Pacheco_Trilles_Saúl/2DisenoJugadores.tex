\chapter{Diseño de jugadores}

\subsection{Características Socioculturales}
Ruku es un juego que a pesar de estar ambientado en una fantasía medieval va ha beber mucho de la cultura del mundo "real". Ya que los niveles del modo campaña esta agrupados en diferentes continentes y estos están ambientados en los continentes del mundo "real". Esto nos permite explorar temas como algún que otro choque cultural o como las habilidades tematizadas en algunos personajes.

Hablando de personajes temáticos, en cada zona habrá mínimo un personaje nativo propio de esta. Este personaje será de uso obligatorio en el continente en el se encuentre. Ya que es este personaje el que esta de una forma u otra luchando por la libertad y resistiéndose ante la opresión de los Slaad en este lugar desde el principio con toda su gente, por lo que es importante que el sera una cara visible a la hora de salvar esa zona. 

Posteriormente dependiendo de como haya ido la narrativa con este personaje se podrá unir a tu \textit{party} (equipo), para los siguientes niveles, continentes e incluso para el apartado multijugador.

%\subsection{Posibles Problemas de Género}
%//Para más tarde TODO
% masc fem




\begin{dragonbox}{Taxonomía de Bartle}
La taxonomía de Bartle es un modelo que clasifica a los jugadores de videojuegos en cuatro tipos principales, basado en sus motivaciones y comportamientos dentro de un entorno de juego. Y cuanto mejor entiendas a tus jugadores, mejor podrás satisfacer sus necesidades. Por lo que vamos a intentar explotar un poco estos tipos buscando que quiere cada jugdor para intentar ofrecérselo.
\begin{itemize}
    \item Achiever: Este jugador busca mostrar su progreso, coleccionando y exhibiendo sus logros e insignias. Por lo que en el juego hay que añadir diferentes logros, coleccionables y lo más importante misiones (sobretodo diarias) para satisfacer a estos jugadores.
    \item Explorer: Los exploradores quieren ver cosas nuevas y descubrir nuevos secretos, para ellos este es el premio, y el máximo exponente en este campo son los huevos de pascua. Elementos nada difíciles de implementar con los que podemos satisfacer a este tipo de jugadores. Ademas en el juego hay distintas zonas (continentes) con distintos mapas lo que da una gran variedad que explorar a la hora de jugar.
    \item Socializer: Este tipo de jugadores representa la gran mayoría. Se caracterizan simplemente por el hecho de poder interactuar entre ellos mismos. Para lograr una mayor audiencia de este tipo, Ruku tiene como ya se ha mencionado anteriormente, un modo de juego multijugador. En este modo de juego te puedes enfrentar a todo tipo de enemigos con los personajes que ya hayas desbloqueado. Como mecánicas secundarias para satisfacer este tipo tenemos: en este modo de juego puedes interactuar con \textit{stikers}, al estilo del \textit{Clash Royal}, o incluso copiar las \textit{builds} de otros jugadores (si tienes todos los requisitos necesarios). Y por ultimo la parte más importante, podrás comprar \textit{debuffs}, es decir cosas que entorpezcan el progreso de tu contrincante durante un tiempo (ya sea se reduce su mana durante x tiempo o aparece algun que otro tipo de enemigo adicional). Esto fomenta enormemente las interacciones entre nuestros jugadores y su competitividad lo que nos lleva al siguiente tipo de jugador.
    %//Como el bloon puedes comprar enemigos para que le jodan
    %// Puedes comprar otros debuffs como le dan menos mana etc
    \item Killer: Los Killers como bien se introduce en el apartado anterior destacan por su fuerte sentimiento de competitividad aunque aquí prima el hecho de que el oponente pierda. Por lo que para fomentar esta competitividad el juego cuenta con un ranking por ligas, cuantos más puntos obtengas (matando a los Slaad) más arriba subiras en las ligas.
\end{itemize}
\end{dragonbox}