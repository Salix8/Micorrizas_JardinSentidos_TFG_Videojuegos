\chapter{Monetización}

\subsection*{Estrategias}
\paragraph{Compras integradas (\textit{In-App Purchases})}: Estas compras serán meramente estéticas. Estas serán skins para personajes, para las armas o incluso efectos especiales para algún ataque, conjuro, partes del perfil.
Por otro lado tenemos la tipica tienda en la que compras piezas de oro (la moneda del juego) y gemas, recordemos que con estas piezas de oro y gemas se puede comerciar con los mercaderes del propio juego y se pude comprar también las \textit{loot boxes} o \textit{stikers} y frases para comunicarte en el modo arena.
\paragraph{Pase de batalla}: Este pase consistirá en recompensas adicionales en el modo arena. Por alcanzar ciertos puntos en el modo arena se dan recompensas pues comprando el pase de batalla adicionalmente entre esas recompensas recibirás mas recompensas.
\paragraph{\textit{Advergaming}}: Estos anuncios consisten en integrar la marca o lo que sea que tenemos que promocionar dentro del videojuego sin interrumpir la experiencia del jugador en este.
\paragraph{Publicidad opcional}: El punto aquí esta en que esto los jugadores lo pueden ver cuando quieran esto quiere decir que si no quieren ver anuncios no obligamos a nadie a que lo vea. Estos anuncios son videos cortos con los cuales obtener piezas de oro o incluso algunas gemas.

\subsection*{Consideraciones de Diseño}
\paragraph{Equilibrio entre pago y progreso}: es esencial asegurar a nuestros jugadores que se puede disfrutar de todo el contenido sin gastarse dinero y que esta sera una experiencia justa y satisfactoria en comparación con la gente que si que se quiera gastar dinero.
\paragraph{Transparencia}: otra de las consideraciones es que debemos ser claros y transparentes a cerca de las ventajas y el valor de los objetos que el jugador va a adquirir en nuestra tienda.

Por otro lado debemos aportar transparencia también a los patrocinadores con los que colaboraremos, ya que esto nos otorgara un estatus de confidencialidad, para futuras colaboraciones y promociones.

\paragraph{Experiencia de usuario}: Otro de los esenciales en Ruku es que queremos evitar las constantes interrupciones para promover las compras. Además nos comprometemos a tener una interfaz intuitiva a la hora de acceder y comprar en la tienda.