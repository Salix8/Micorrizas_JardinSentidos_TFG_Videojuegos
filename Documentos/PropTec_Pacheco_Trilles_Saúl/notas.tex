\begin{enumerate}
    \item \textbf{padre}
    \begin{itemize}
        \item \textbf{hijo}: 
        \item \textbf{hijo2}: 
    \end{itemize}
    \item \textbf{padre2}
    \begin{itemize}
        \item \textbf{hijo}: 
        \item \textbf{hijo2}: 
    \end{itemize}
\end{enumerate}

Pagina en horizontal 
\begin{landscape}
\end{landscape}

Bibliografia 

https://www.interaction-design.org/literature/article/bartle-s-player-types-for-gamification


\begin{enumerate}
    \item \textbf{Mecánicas de los personajes}: 
    \begin{itemize}
        \item \textbf{Curso de acción}: 
        \item \textbf{Personalización}: 
        \vspace{0.1cm}
    \end{itemize}
    \item \textbf{Mecánicas de los NPCs}: 
    \begin{itemize}
        \item \textbf{Interacciones}: (Alianzas) 
        \item \textbf{Comportamientos}: (IA patrones)
    \end{itemize}
    \item \textbf{Mecánicas core}:
    \begin{itemize}
        \item \textbf{Sistema de combate}: 
        \item \textbf{Economía de Acción}: (Gestión de Recursos )
    \end{itemize}
\end{enumerate}

\makecell{int mov \\ bool tDifícil \\ int daño}



Para desarrollar la estructura de retos en tu juego "Ruku: la caída de un mundo", podemos desglosar los desafíos en tres niveles: retos atómicos, retos intermedios y el reto final. Aquí te proporciono una guía sobre cómo podrías estructurarlos:a. Desarrollo de la estructura de retosi. Retos atómicosLos retos atómicos son los desafíos más básicos y frecuentes que los jugadores encontrarán durante el juego. Estos pueden incluir:Enemigos comunes: Enfrentamientos con enemigos básicos que requieren el uso de habilidades simples y estrategias rápidas. Estos encuentros ayudan a los jugadores a familiarizarse con las mecánicas de combate por turnos y el uso de acciones grandes, pequeñas y gratuitas.Puzzles simples: Pequeños acertijos o tareas que los jugadores deben resolver para avanzar, como activar un mecanismo o encontrar un objeto oculto.Recolección de recursos: Tareas que implican recolectar materiales necesarios para mejorar el equipo o las habilidades de los personajes.ii. Retos intermediosLos retos intermedios son más complejos y requieren una planificación estratégica más profunda. Estos pueden incluir:Mini-jefes: Enemigos más fuertes que los comunes, que requieren el uso coordinado de habilidades de equipo y una gestión cuidadosa de los recursos.Misiones secundarias: Tareas opcionales que ofrecen recompensas adicionales, como experiencia, equipo o habilidades especiales. Estas misiones pueden involucrar múltiples pasos y decisiones que afectan el desarrollo del juego.Eventos de defensa: En el modo campaña, los jugadores pueden enfrentar oleadas de enemigos que intentan invadir una zona específica, similar al estilo de "plantas versus zombies ". Estos eventos requieren una estrategia defensiva sólida.iii. Reto finalEl reto final es el clímax del juego, donde los jugadores deben aplicar todo lo que han aprendido. Este puede incluir:Batalla contra el jefe final: Un enfrentamiento épico contra un enemigo poderoso, como el líder de los Slaads . Esta batalla debe ser desafiante y requerir el uso de todas las habilidades y estrategias disponibles.Decisión narrativa crucial: Una elección importante que afecta el desenlace de la historia, ofreciendo diferentes finales basados en las decisiones del jugador a lo largo del juego.Desafío de resistencia: Una serie de encuentros consecutivos sin la posibilidad de descansar o reabastecerse, poniendo a prueba la gestión de recursos y la resistencia del equipo del jugador.





Elementos de Aleatoriedad:
Cofres con recompensas aleatorias que incrementan la expectativa y el interés.
Eventos dinámicos dentro del juego que ofrecen oportunidades únicas para obtener beneficios especiales.

Desafíos Incrementales:
Niveles de dificultad ajustables que se adaptan a la curva de aprendizaje del jugador.
Introducción gradual de nuevas mecánicas y enemigos para mantener la experiencia fresca y desafiante.

Feedback Inmediato:
Indicadores visuales y sonoros que celebran los éxitos del jugador.
Respuestas rápidas a las acciones realizadas, como efectos especiales tras realizar un ataque crítico o resolver un puzzle.


Multijugador
los humanos son muy variados
Como se juega??

Competitivo
Cooperativo

Cuantos jugadores pueden jugar de 1 a 4

Online (no tiene sentido un multijugador local)

Como interactúan

Narrativa integrada??

las mecánicas no cambian pasas de que un personaje controle 4 personajes a que solo pueda controlar uno.

Progresión

Veteranos y novatos

Economía, los jugadores como tal no pueden 



https://jesusjmuji.github.io/recursos-uji/