\chapter{Multijugador}
%El modo multijugador es un componente con mucho potencial para aumentar la rejugabilidad y competencia en Ruku.

\subsection*{Modo arena}
El modo Arena es el centro de la experiencia multijugador competitiva. Los jugadores se enfrentarán en partidas idénticas que pondrán a prueba su habilidad para optimizar estrategias tanto antes de empezar a jugar como una vez ya en el campo de batalla bajo presión.
\paragraph{Mecánicas del juego}: 
\begin{itemize}
    \item \textbf{Preparaciones}: Antes de darle a jugar a este modo puedes preparar a tus héroes con los conjuros, habilidades y objetos que prefieras.
    \item \textbf{Comenzar}: Al inicio tendrás unos segundos para editar la disposición de los héroes, sus habilidades y objetos. Pero, ¿Porque se hace esto? Bien pues esto es porque en esta fase se te mostrara el tipo de enemigos contra los que te vas a enfrentar. Esto supone que si tienes mejores habilidades contra ciertos tipos de Slaads seguramente aproveches para cambiarte a esas habilidades para así obtener más puntos que tu adversario y subir en las ligas.
    \item \textbf{Salida al campo de batalla}: Para esto también tendrás unos segundos antes de que empiece la invasión, esta fase consiste en posicionar a tus personajes en el campo de batalla, la idea es posicionarlos estratégicamente sabiendo las coberturas la dirección en la que viene el enemigo y por supuesto como ya hemos visto en la fase anterior los enemigos en especifico. En caso de no haber desplegado los personajes a tiempo se colocaran en una posición aleatoria en todo el mapa.
    \item \textbf{\textit{Matchmaking}}: \textit{Matchmaking} es un sistema de emparejamiento basado en el nivel y la liga del jugador, asegurando partidas justas y desafiantes para todos los jugadores. Ganar y perder a partes iguales es satisfactorio ganar todo el rato y perder todo el rato ya no tanto.
    \item \textbf{Invasión}: Con los personajes ya posicionados empieza el combate. Al igual que en el resto del juego se tira iniciativa y cada uno hace su turno. Y la invasión dura un tiempo en el que los enemigos van saliendo progresivamente.
    \item \textbf{Oleada Cruzada}: Los jugadores pueden gastar monedas Sladdi que se obtienen en las recompensas del modo arena, para enviar enemigos adicionales al oponente y con esto dificultar su progreso incluso hacerle perder personajes por lo que hacer casi imposible que le supere. Esta moneda obviamente no se podrá comprar, ya que si no infringiríamos las consideraciones que hemos tenido en cuenta y dicho que cumpliríamos en el capitulo de la monetización.
    \item \textbf{Puntuaciones}: Cuando los personajes hacen daño a un enemigo el jugador recibe puntos. Si el personaje hace \textit{"over damage"}, es decir, el enemigo tiene 5 de vida y el personaje hace 13 de daño pues eso que hace de más ese exceso de daño se duplicara en puntos. Por lo que en este caso el jugador recibe 5 puntos por la vida del enemigo más el exceso de daño por dos en este caso 8*2 que son 16 puntos, en total 21 puntos.
    
    Nota: los enemigos generados por el oponente no tienen el bonificador del *2 en el exceso de daño.
    \item \textbf{Ligas}: El ganador es quien haya obtenido más puntos (estos se acaban de explicar en el apartado anterior). Este jugador que ha ganado subirá posiciones en la liga, mientras que el otro jugador se mantendrá en la posición que estaba antes de jugar esta partida (por lo que no cambiara nada para el, solo habrá adquirido experiencia de juego). ¿Porque hacemos esto? Sencillo, si no hay un castigo por perder no hay miedo a jugar, hay jugadores que dicen estoy en una buena posición no me quiero arriesgar a perder y eso puede poner en riego la fidelización del jugador a Ruku, no debemos olvidar que cada mes las ligas se reiniciarán.
    \item \textbf{Comunicación}: Durante el tiempo que se esta en el campo de batalla y después mientras se muestra quien ha conseguido una mayor puntuación los jugadores tendrás disponible un botón para comunicarse con su oponente. En este botón te saldrán diversas opciones, las cuales podrás definir de antemano. Estas opciones varían entre \textit{stikers} y frases.
    \item \textbf{Recompensas}: A medida que se va subiendo en las ligas obtienes diferentes recompensas oro, gemas... Pero esencialmente lo que los jugadores buscaran conseguir son las monedas Slaadi que como se ha mencionado anteriormente estas monedas sirven para en medio de una invasión mandar mas Slaads a tus oponentes y así dificultarles su escalada a la victoria y quedarte tu con dicha victoria.
\end{itemize}