\chapter{Gameplay}

\subsection*{Objetivo principal}
El objetivo principal en Ruku es la defensa del mundo, el jugador controla diferentes aventureros los cuales utilizando sus habilidades podrán defenderse a ellos mismos incluso podrán \textbf{salvar el mundo} de la invasión Slaad.

\subsection*{Retos principales}
\begin{itemize}
    \item \textbf{Administrar recursos}: el dinero es el recurso básico en el que se piensa siempre pero en Ruku como muchos otros RPGs tienes que elegir también que aventureros llevar a la batalla y esto quieras o no es un recurso a administrar estratégicamente.
    \item \textbf{Adaptabilidad}: Adaptarse a los diferentes tipos tanto de escenarios y sobre todo a los diferentes tipos de Slaads y sus resistencias. Cuando te has adaptado a luchar contra un tipo de Slaad empiezan a venir otro tipo de Slaads los que te hacen cambiar toda la estrategia.
    \item \textbf{Niveles}: Hay que superar los niveles de cada continente y cada continente tiene una temática con ciertas mecánicas.
\end{itemize}
    
\subsection*{Subretos}
\begin{itemize}
    \item \textbf{Condiciones}: No todos los niveles son: "mata a estos Slaads". Hay algunos, que tienen desafíos a parte de tener que derrotar a los Slaads que te atacan. Por ejemplo te dicen solo puedes coger aventureros que sean devotos, solo puedes coger guerreros, superar el nivel en x tiempo, no gastes más de 5 conjuros, etc.
    \item \textbf{Desafíos opcionales}: Los desafíos opcionales esta claro lo que son, hay una condición que no implica que el jugador gane o pierda el nivel, pero si realiza este desafío si que le da un extra, por ejemplo en monedas o en un nivel secreto como se ha mencionado anteriormente.
    \item \textbf{\textit{Bosses}}: El último nivel de cada continente es más difícil que el resto de niveles del continente, ya que este contiene un \textit{boss} (de los que ya hemos hablado como funcionan en las reglas) a parte de un mapa de combate especifico para este.
    \item \textbf{Entorno}: Aprovechar las mecánicas y el entorno de cada continente, ya que estos son muy polivalentes y se pueden usar a tu favor, claro que también los podrán usar en tu contra, como se ha comentado anteriormente también.
\end{itemize}

\subsection*{Retos atómicos}
\begin{itemize}
    %\item \textbf{Supervivencia}: Que tu aventurero siga en pie es importante para superar el nivel, así que evitar que te hagan daño es una buena idea. 
    \item \textbf{Acciones}: Las acciones cuando las gastas puede que la puedas ejecutar o no. Por ejemplo la acción de ataque, el resultado es atómico ya que solo hay dos resultados posibles le ha dado al objetivo o no, si es que si le hago el daño pertinente, si es que no pues no pasa nada. Entre otros ejemplos están, desarmar trampas, usar un objeto, lanzar conjuro, empujar, \textit{lootear}, etc.
    \item \textbf{Reclutar}: Si el jugador quiere puede reclutar a los NPCs lo cual es una  decisión totalmente libre y el jugador quien decide si ir acompañado o no.
\end{itemize}

\begin{figure}
  \centering
  \includegraphics[width=0.7\textwidth]{img/GameplayRuku.png}
  \caption{Esquema }
  \label{fig:EsqGameplay}
\end{figure}